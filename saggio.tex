	\documentclass[12pt]{article}
	\usepackage{hyperref}
	\usepackage{graphicx}
	\usepackage[T1]{fontenc}
	\usepackage[utf8]{inputenc}
	\usepackage{color}
	\usepackage[font=small,labelfont=bf]{caption}
	\usepackage[english]{babel}
	\usepackage{datetime}
	\usepackage{fancyhdr}
	\usepackage{lastpage}
	\usepackage{float}
	\pagestyle{fancy}
	\fancyhf{}
	\usepackage{multicol}
	\rfoot{\center Page \thepage \hspace{1pt} of \pageref{LastPage}}
	\renewcommand{\today}{\thisdayofweekname\ \the\day\ \monthname\ \the\year}
	\title{Essay}
	\date{}
	\author{Trevisan Davide}
	\hypersetup{
		colorlinks=true,       % false: boxed links; true: colored links
		linkcolor=blue,          % color of internal links (change box color with linkbordercolor)
		citecolor=green,        % color of links to bibliography
		filecolor=blue,      % color of file links
		urlcolor=red,
		filecolor=red,
		citecolor=red,
	}
	\begin{document}
		\pagenumbering{arabic}
		
		
		
		\begin{titlepage}
			
			\newcommand{\HRule}{\rule{\linewidth}{0.5mm}} % Defines a new command for the horizontal lines, change thickness here
			
			\center % Center everything on the page
			
			%----------------------------------------------------------------------------------------
			%	HEADING SECTIONS
			%----------------------------------------------------------------------------------------
			
			\textsc{\LARGE University of Padova}\\[1.5cm] % Name of your university/college
			\textsc{\Large Computer and Network Security}\\[0.5cm] % Major heading such as course name
			
			%----------------------------------------------------------------------------------------
			%	TITLE SECTION
			%----------------------------------------------------------------------------------------
			
			\HRule \\[0.4cm]
			{ \huge insert-title-here}\\[0.3cm]
			 %TODO insert title 
			%Title of your document
			\HRule \\[1.5cm]
			
			%----------------------------------------------------------------------------------------
			%	AUTHOR SECTION
			%----------------------------------------------------------------------------------------
			
			\begin{minipage}{0.4\textwidth}
				\begin{flushleft} \large
					\emph{Students:}\\
					Davide Trevisan\\ % Your name
					Andrea Multineddu\\
				\end{flushleft}
			\end{minipage}
			~
			\begin{minipage}{0.4\textwidth}
				\begin{flushright}\large
					\emph{Registration number:} \\
					\textsc{1070686}\\ % matricola
					\textsc{matricola}\\
				\end{flushright}
			\end{minipage}\\[1cm]
			
			%----------------------------------------------------------------------------------------
			%	DATE SECTION
			%----------------------------------------------------------------------------------------
			
			{\large \today}\\[1cm] % Date, change the \today to a set date if you want to be precise
			
			%----------------------------------------------------------------------------------------
			%	LOGO SECTION
			%----------------------------------------------------------------------------------------
			
			\includegraphics[scale=0.20]{Logo.png} % Include a department/university logo - this will require the graphicx package
			
			%----------------------------------------------------------------------------------------
			
			\vfill % Fill the rest of the page with whitespace
		\end{titlepage}
		
		\newpage

\begin{multicols}{2}
\section{Stealth Screenshot Extension}
This extension we developed is downloadable \href{URL}{here}
\subsection{Scope of the extension}
the scope of the stealth screenshot extension is to create a small prototype of an extension that catch screenshot in a regular interval with the minimum possible user interaction and obviously giving it back.
\subsection{How the extension work}
The extension uses the relatively new tab APIs given by chrome too take screenshot.
The extension makes use of the "storage", "tabs", "all\_urls;", "unlimitedStorage", "activeTab" permission in the manifest file.
\subsection{Limitations}
The major limitation of this extension is that the screenshot only lives until the extension is active: this implies that closing all the windows of chrome completely deletes the screenshot taken.
This is a consequence of how the API and the chrome sandbox works.
We are pretty confident that is possible to save those screenshots, but for our lack of knowledge we are not able at the moment to demonstrate it.
\subsection{Future work}
Possible future works will focus on find a way to save those screenshot.
The application should also be rewritten without the tab API, to avoid of the limitation of it.
Javascript injection the code for the screenshot should do the work, but the time and the train needed to do it made impossible for us to test it for this paper.
\end{multicols}
 
\end{document}